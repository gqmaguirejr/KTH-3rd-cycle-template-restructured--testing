%\NeedsTeXFormat{LaTeX2e}[1999/12/01]
%\ProvidesPackage{kth-commands}
%    [2025/07/19 v0.1 Provides template-specific commands needed for kththesis.cls]

\makeatletter
%% Macros from the package phfnote
% \phfnoteSaveDefs{〈identifier〉}{〈list of macro names〉} saves the current definitions of the given list of macros under the identifier. The list of macros is specified as a comma-separated list of macro names. Note that there cannot be spaces after the commas in the list.

\def\phfnoteSaveDefs#1#2{%
  \csgdef{phfnote@restoredefs@#1}{}%
  \def\@tmpa{#2}%
  \@for\next:=\@tmpa\do{%
    \global\csletcs{phfnote@restoredefs@#1@\next}{\next}%
    \expandafter\xappto\csname phfnote@restoredefs@#1\endcsname{%
      \noexpand\csletcs{\next}{phfnote@restoredefs@#1@\next}%
    }%
  }%
}

% \phfnoteRestoreDef{〈identifier〉} - restores the saved macros
\def\phfnoteRestoreDefs#1{%
  \ifcsname phfnote@restoredefs@#1\endcsname%
    \csname phfnote@restoredefs@#1\endcsname%
  \else%
    \PackageError{kththesis}{\string\phfnoteRestoreDefs: no such
      definitions stored (#1)}{}
  \fi%
}
\makeatother

%% To give warnings about the use of citations, we set a hook before the abstract and one after it
%% The first hook saves the definition of \cite and sets up a warning while the second hook restores it (similarly for the other commands that should not be used in an abstract
\BeforeBeginEnvironment{abstract}{\phfnoteSaveDefs{origcmds}{cite,ref,Cref,cref}%
\renewcommand{\cite}[1]{{\NotoSansFont \textcolor{red}{⁉\uuline{cite}⁉}}\PackageWarning{kththesis}{attempt to use cite command in an abstract}}%
\renewcommand{\ref}[1]{{\NotoSansFont \textcolor{red}{⁉\uuline{ref}⁉}}\PackageWarning{kththesis}{attempt to use ref command in an abstract}}%
\renewcommand{\Cref}[1]{{\NotoSansFont \textcolor{red}{⁉\uuline{Cref}⁉}}\PackageWarning{kththesis}{attempt to use Cref command in an abstract}}%
\renewcommand{\cref}[1]{{\NotoSansFont \textcolor{red}{⁉\uuline{cref}⁉}}\PackageWarning{kththesis}{attempt to use cref command in an abstract}}%
}

\makeatletter
% Define commands for setting the user-definable parts of the title page

\let\@subtitle\@empty
\newcommand{\subtitle}[1]{\def\@subtitle{#1}}
\let\@alttitle\@empty
\newcommand{\alttitle}[1]{\def\@alttitle{#1}}
\let\@altsubtitle\@empty
\newcommand{\altsubtitle}[1]{\def\@altsubtitle{#1}}

% enable the author to specify plain text titles and subtitles to handle the situation when the titles and subtitles cannot be translated to pure Unicode

\let\@titleInPlainText\@empty
\newcommand{\titleInPlainText}[1]{\def\@titleInPlainText{#1}}
\let\@subtitleInPlainText\@empty
\newcommand{\subtitleInPlainText}[1]{\def\@subtitleInPlainText{#1}}
\let\@alttitleInPlainText\@empty
\newcommand{\alttitleInPlainText}[1]{\def\@alttitleInPlainText{#1}}
\let\@altsubtitleInPlainText\@empty
\newcommand{\altsubtitleInPlainText}[1]{\def\@altsubtitleInPlainText{#1}}


\let\@hostcompany\@empty
\newcommand{\hostcompany}[1]{\def\@hostcompany{#1}}
\let\@hostorganization\@empty
\newcommand{\hostorganization}[1]{\def\@hostorganization{#1}}

\let\@school\@empty
\newcommand{\school}[1]{\def\@school{#1}}

% First author's information
\let\@authorsLastname\@empty
\newcommand{\authorsLastname}[1]{\def\@authorsLastname{#1}}
\let\@authorsFirstname\@empty
\newcommand{\authorsFirstname}[1]{\def\@authorsFirstname{#1}}
\let\@email\@empty
\newcommand{\email}[1]{\def\@email{#1}}
\let\@kthid\@empty
\newcommand{\kthid}[1]{\def\@kthid{#1}}
\let\@orcid\@empty
\newcommand{\orcid}[1]{\def\@orcid{#1}}
\let\@authorsSchool\@empty
\newcommand{\authorsSchool}[1]{\def\@authorsSchool{#1}}
\let\@authorsDepartment\@empty
\newcommand{\authorsDepartment}[1]{\def\@authorsDepartment{#1}}
% Information about the city and country for the acknowledgement
\let\@authorCity\@empty
\newcommand{\authorCity}[1]{\def\@authorCity{#1}}
\let\@authorCountry\@empty
\newcommand{\authorCountry}[1]{\def\@authorCountry{#1}}
\let\@authorCityCountryDate\@empty
% Define the command so that if a date is not specified it will not output the comma; it only outputs the comma if there is a date.
\ExplSyntaxOn
\cs_set_eq:NN  \IfEmptyTF  \tl_if_blank:nTF
\ExplSyntaxOff
\NewDocumentCommand{\authorCityCountryDate}{m}{
\def\@authorCityCountryDate{\ifx\@authorCity\@empty%
    Stockholm%
    \else%
    \ifx\@authorCountry\@empty%
    \@authorCity%
    \else%
    \@authorCity, \@authorCountry%
    \fi%
    \fi%
    \IfEmptyTF{#1}{}{, #1}%
    }%
}
\authorCityCountryDate{\MONTH\enspace\the\year} % default month and year for the acknowledgement signature

% Second author's information
\let\@secondAuthorsLastname\@empty
\newcommand{\secondAuthorsLastname}[1]{\def\@secondAuthorsLastname{#1}}
\let\@secondAuthorsFirstname\@empty
\newcommand{\secondAuthorsFirstname}[1]{\def\@secondAuthorsFirstname{#1}}
\let\@secondemail\@empty
\newcommand{\secondemail}[1]{\def\@secondemail{#1}}
\let\@secondkthid\@empty
\newcommand{\secondkthid}[1]{\def\@secondkthid{#1}}
\let\@secondorcid\@empty
\newcommand{\secondorcid}[1]{\def\@secondorcid{#1}}
\let\@secondAuthorsSchool\@empty
\newcommand{\secondAuthorsSchool}[1]{\def\@secondAuthorsSchool{#1}}
\let\@secondAuthorsDepartment\@empty
\newcommand{\secondAuthorsDepartment}[1]{\def\@secondAuthorsDepartment{#1}}
% Information about the city and country for the acknowledgement
\let\@secondAuthorCity\@empty
\newcommand{\secondAuthorCity}[1]{\def\@secondAuthorCity{#1}}
\let\@secondAuthorCountry\@empty
\newcommand{\secondAuthorCountry}[1]{\def\@secondAuthorCountry{#1}}
\let\@secondAuthorCityCountryDate\@empty
% Define the command so if a date is not specified it will not output the comma; it only outputs the comma if there is a date.
\NewDocumentCommand{\secondAuthorCityCountryDate}{m}{
\def\@secondAuthorCityCountryDate{\ifx\@secondAuthorCity\@empty%
    Stockholm%
    \else%
    \ifx\@secondAuthorCountry\@empty%
    \@secondAuthorCity%
    \else%
    \@secondAuthorCity, \@secondAuthorCountry%
    \fi%
    \fi%
    \IfEmptyTF{#1}{}{, #1}%%
    }%
}

\let\@supervisorAsLastname\@empty
\newcommand{\supervisorAsLastname}[1]{\def\@supervisorAsLastname{#1}}
\let\@supervisorAsFirstname\@empty
\newcommand{\supervisorAsFirstname}[1]{\def\@supervisorAsFirstname{#1}}
\let\@supervisorAsEmail\@empty
\newcommand{\supervisorAsEmail}[1]{\def\@supervisorAsEmail{#1}}
\let\@supervisorAsKTHID\@empty
\newcommand{\supervisorAsKTHID}[1]{\def\@supervisorAsKTHID{#1}}
\let\@supervisorAsOrganization\@empty
\newcommand{\supervisorAsOrganization}[1]{\def\@supervisorAsOrganization{#1}}
\let\@supervisorAsSchool\@empty
\newcommand{\supervisorAsSchool}[1]{\def\@supervisorAsSchool{#1}}
\let\@supervisorAsDepartment\@empty
\newcommand{\supervisorAsDepartment}[1]{\def\@supervisorAsDepartment{#1}}

\let\@supervisorBsLastname\@empty
\newcommand{\supervisorBsLastname}[1]{\def\@supervisorBsLastname{#1}}
\let\@supervisorBsFirstname\@empty
\newcommand{\supervisorBsFirstname}[1]{\def\@supervisorBsFirstname{#1}}
\let\@supervisorBsEmail\@empty
\newcommand{\supervisorBsEmail}[1]{\def\@supervisorBsEmail{#1}}
\let\@supervisorBsKTHID\@empty
\newcommand{\supervisorBsKTHID}[1]{\def\@supervisorBsKTHID{#1}}
\let\@supervisorBsOrganization\@empty
\newcommand{\supervisorBsOrganization}[1]{\def\@supervisorBsOrganization{#1}}
\let\@supervisorBsSchool\@empty
\newcommand{\supervisorBsSchool}[1]{\def\@supervisorBsSchool{#1}}
\let\@supervisorBsDepartment\@empty
\newcommand{\supervisorBsDepartment}[1]{\def\@supervisorBsDepartment{#1}}

\let\@supervisorCsLastname\@empty
\newcommand{\supervisorCsLastname}[1]{\def\@supervisorCsLastname{#1}}
\let\@supervisorCsFirstname\@empty
\newcommand{\supervisorCsFirstname}[1]{\def\@supervisorCsFirstname{#1}}
\let\@supervisorCsEmail\@empty
\newcommand{\supervisorCsEmail}[1]{\def\@supervisorCsEmail{#1}}
\let\@supervisorCsKTHID\@empty
\newcommand{\supervisorCsKTHID}[1]{\def\@supervisorCsKTHID{#1}}
\let\@supervisorCsOrganization\@empty
\newcommand{\supervisorCsOrganization}[1]{\def\@supervisorCsOrganization{#1}}
\let\@supervisorCsSchool\@empty
\newcommand{\supervisorCsSchool}[1]{\def\@supervisorCsSchool{#1}}
\let\@supervisorCsDepartment\@empty
\newcommand{\supervisorCsDepartment}[1]{\def\@supervisorCsDepartment{#1}}

\let\@supervisorDsLastname\@empty
\newcommand{\supervisorDsLastname}[1]{\def\@supervisorDsLastname{#1}}
\let\@supervisorDsFirstname\@empty
\newcommand{\supervisorDsFirstname}[1]{\def\@supervisorDsFirstname{#1}}
\let\@supervisorDsEmail\@empty
\newcommand{\supervisorDsEmail}[1]{\def\@supervisorDsEmail{#1}}
\let\@supervisorDsKTHID\@empty
\newcommand{\supervisorDsKTHID}[1]{\def\@supervisorDsKTHID{#1}}
\let\@supervisorDsOrganization\@empty
\newcommand{\supervisorDsOrganization}[1]{\def\@supervisorDsOrganization{#1}}
\let\@supervisorDsSchool\@empty
\newcommand{\supervisorDsSchool}[1]{\def\@supervisorDsSchool{#1}}
\let\@supervisorDsDepartment\@empty
\newcommand{\supervisorDsDepartment}[1]{\def\@supervisorDsDepartment{#1}}

\let\@supervisorEsLastname\@empty
\newcommand{\supervisorEsLastname}[1]{\def\@supervisorEsLastname{#1}}
\let\@supervisorEsFirstname\@empty
\newcommand{\supervisorEsFirstname}[1]{\def\@supervisorEsFirstname{#1}}
\let\@supervisorEsEmail\@empty
\newcommand{\supervisorEsEmail}[1]{\def\@supervisorEsEmail{#1}}
\let\@supervisorEsKTHID\@empty
\newcommand{\supervisorEsKTHID}[1]{\def\@supervisorEsKTHID{#1}}
\let\@supervisorEsOrganization\@empty
\newcommand{\supervisorEsOrganization}[1]{\def\@supervisorEsOrganization{#1}}
\let\@supervisorEsSchool\@empty
\newcommand{\supervisorEsSchool}[1]{\def\@supervisorEsSchool{#1}}
\let\@supervisorEsDepartment\@empty
\newcommand{\supervisorEsDepartment}[1]{\def\@supervisorEsDepartment{#1}}


\let\@courseCycle\@empty
\newcommand{\courseCycle}[1]{\def\@courseCycle{#1}}
\let\@courseCode\@empty
\newcommand{\courseCode}[1]{\def\@courseCode{#1}}
\let\@courseCredits\@empty
\newcommand{\courseCredits}[1]{\def\@courseCredits{#1}}

\let\@degreeName\@empty
\newcommand{\degreeName}[1]{\def\@degreeName{#1}}

% To support Doctor of Philosophy and Licentiate of Philosophy degrees in addition to Tekn. Dr. and Tekn. Lic.
\let\@degreeModifier\@empty
\newcommand{\degreeModifier}[1]{\def\@degreeModifier{#1}}

\let\@subjectArea\@empty
\newcommand{\subjectArea}[1]{\def\@subjectArea{#1}}

\let\@secondDegreeName\@empty
\newcommand{\secondDegreeName}[1]{\def\@secondDegreeName{#1}}
\let\@secondSubjectArea\@empty
\newcommand{\secondSubjectArea}[1]{\def\@secondSubjectArea{#1}}

% Data related to the oral presentation
\let\@presentationDateAndTimeISO\@empty
\newcommand{\presentationDateAndTimeISO}[1]{\def\@presentationDateAndTimeISO{#1}}
\let\@presentationLanguage\@empty
\newcommand{\presentationLanguage}[1]{\def\@presentationLanguage{#1}}
\let\@presentationRoom\@empty
\newcommand{\presentationRoom}[1]{\def\@presentationRoom{#1}}
\let\@presentationAddress\@empty
\newcommand{\presentationAddress}[1]{\def\@presentationAddress{#1}}
\let\@presentationCity\@empty
\newcommand{\presentationCity}[1]{\def\@presentationCity{#1}}

\let\@opponentsNames\@empty
\newcommand{\opponentsNames}[1]{\def\@opponentsNames{#1}}

% Data for DIVA National Subject Cateories fields
\let\@nationalsubjectcategories\@empty
\newcommand{\nationalsubjectcategories}[1]{\def\@nationalsubjectcategories{#1}}

% Data for UN's Sustainable Development Goals (SDGs)
\let\@SDGs\@empty
\newcommand{\SDGs}[1]{\def\@SDGs{#1}}

% Keywords
\let\@EnglishKeywords\@empty
\newcommand{\EnglishKeywords}[1]{\def\@EnglishKeywords{#1}}

\let\@SwedishKeywords\@empty
\newcommand{\SwedishKeywords}[1]{\def\@SwedishKeywords{#1}}

\ExplSyntaxOn
\cs_new_eq:NN \strcompare \str_if_eq:nnTF

%expand first argument before doing the comparison
\cs_new_eq:NN \strcompareV \str_if_eq:VnTF
\ExplSyntaxOff


%% Changed to use expl3 strcompare rather than xstring's IfEqCase, since the later is not expandable
%% The insight for this is from Enrico Gregorio - egreg's posing of 26 April 2016 at 
%% https://tex.stackexchange.com/questions/306484/how-do-i-perform-an-expandable-string-comparison

\newcommand{\InsertKeywords}[1]{
\strcompare{#1}{english}{\@EnglishKeywords}{}%
\strcompare{#1}{swedish}{\@SwedishKeywords}{}%
}

\let\@programcode\@empty
\newcommand{\programcode}[1]{%
\def\@programcode{#1}%
\edef\@prgmcode{\programmecodeToString{#1}}}

\let\@secondProgramcode\@empty
\newcommand{\secondProgramcode}[1]{%
\def\@secondProgramcode{#1}%
\edef\@secondprgmcode{\programmecodeToString{#1}}}

\let\@edprogram\@empty
\newcommand{\edprogram}[1]{%
\def\@edprogram{#1}}

\let\@secondedProgram\@empty
\newcommand{\secondedProgram}[1]{%
\def\@secondedProgram{#1}}

%% For dealing with the 3rd cycle education subjects
\let\@educationSubjectcode\@empty
\newcommand{\educationSubjectcode}[1]{%
\def\@educationSubjectcode{#1}%
}

% to store the user's choice of copyright or copyleft
\let\@copyrightleft\@empty
% note that the command has to change the definition of \@copyrightleft globally, hence \gdef and not \def
\newcommand{\thesiscopyrightleft}[1]{\gdef\@copyrightleft{#1}}


% Get and store information about the series and the number within this series, i.e, TRITA numbers
%"Series": \{
%	"Title of series": "TRITA-ICT-EX",
%	"No. in series": "2019:00"
\let\@thesisSeries\@empty
\let\@thesisSeriesNumber\@empty
\newcommand{\trita}[2]{\def\@thesisSeries{#1}\def\@thesisSeriesNumber{#2}}

\let\@thesisISBN\@empty
\newcommand{\thesisISBN}[1]{\def\@thesisISBN{#1}}

% for cover illustration
\let\@coverIllustration\@empty
\newcommand{\coverIllustration}[1]{\def\@coverIllustration{#1}}
\let\@coverIllustrationCredit\@empty
\newcommand{\coverIllustrationCredit}[1]{\def\@coverIllustrationCredit{#1}}

\let\@printedBy\@empty
\newcommand{\printedBy}[1]{\def\@printedBy{#1}}

\let\@defenseDescription\@empty
\newcommand{\defenseDescription}[1]{\def\@defenseDescription{#1}}
\makeatother

% Due to parsing problems with Overleaf's checking for balance of
% start and ends of beginning and ending use of environments and
% because babel and polyglossia & biblatex use different names for Norsk and German - this is _not_ automatically supported

% --- Helper macros for language environment switching ---
\ifluatex
\makeatletter
\newcommand{\babelpolyLangStart}[1]{%
    \selectlanguage{#1}% Original definition
    % Output the current values of the font variables to the log file
    %\typeout{babelpolyLangStart #1 \f@encoding\  \f@family\  \f@series\  \f@shape}%
}
% Make @ a non-letter again
\makeatother
\else
    \newcommand{\babelpolyLangStart}[1]{\selectlanguage{#1}}
\fi

\ifluatex
\makeatletter
    \newcommand{\babelpolyLangStop}[1]{%
    %\typeout{babelpolyLangStop #1 \f@encoding\  \f@family\  \f@series\  \f@shape}%
}
% Make @ a non-letter again
\makeatother

\else
    \newcommand{\babelpolyLangStop}[1]{}
\fi

\ExplSyntaxOn
\NewDocumentCommand{\replaceBS}{mm}
 {
  \tl_set_eq:NN #2 #1
  \tl_replace_all:NVn #2 \c_backslash_other_tl { ¢ }
 }
\tl_const:Nx \c_backslash_other_tl { \cs_to_str:N \\ }
\cs_generate_variant:Nn \tl_replace_all:Nnn { NV }
\ExplSyntaxOff

\makeatother


% set up dividerContent to have smaller margins for the content on Divider pages
\def\dividerContent#1#2{\list{}{\rightmargin#2\leftmargin#1}\item[]}
\let\enddividerContent=\endlist 

% Define some functions to make it easier to open, close, and write to the file if it is open, while generating errors if the file is not open
\newcommand{\FileOpen}[2]{%
% arg 1 is a filehandle as a string
% arg 2 is a filename
    \expandafter\newwrite\csname#1\endcsname%
    \expandafter\immediate\expandafter\openout\csname#1\endcsname #2%
}

\newcommand{\FileClose}[1]{%
% arg 1 is a filehandle as a string
    \expandafter\immediate\expandafter\closeout\csname#1\endcsname%
    \expandafter\let\csname#1\endcsname\undefined % Mark the handle as undefined after closing
}


\newcommand{\WriteIfFileOpen}[2]{%
% arg 1 is a filehandle
% arg 2 is content to be written to a file
  \expandafter\ifcsname\string#1\endcsname
    % The control sequence for the file handle exists
    \immediate\write\csname #1\endcsname{#2}%
  \else
    % The control sequence for the file handle is not defined
    \PackageWarning{kththesis}{File handle '#1' is not defined (or closed).}%
  \fi
}


% for dealing with the List of publications
% Note that ragged right has been used to avoid hyphenation issues in the titles
% uses functionality of enumitem package
\newlist{ListOfPapers}{enumerate}{1}
\setlist[ListOfPapers]{
    leftmargin=*,
    label={Paper \Alph*},
    ref={Paper \Alph*},
    resume=listOFPapers,
    itemsep=0em,
    first=\raggedright
  }

\newlist{ListOfPosters}{enumerate}{1}
\setlist[ListOfPosters]{
    leftmargin=*,
    label={Poster \Alph*},
    ref={Poster \Alph*},
    resume=ListOfPosters,
    itemsep=0em,
    first=\raggedright
  }

\newlist{ListOfPatents}{enumerate}{1}
\setlist[ListOfPatents]{
    leftmargin=*,
    label={Patent \Alph*},
    ref={Patent \Alph*},
    resume=ListOfPatents,
    itemsep=0em,
    first=\raggedright
  }

\newlist{ListOfPatentApplications}{enumerate}{1}
\setlist[ListOfPatentApplications]{
    leftmargin=*,
    label={Patent Application \Alph*},
    ref={Patent Application \Alph*},
    resume=ListOfPatentApplications,
    itemsep=0em,
    first=\raggedright
  }

\newlist{ListOfReports}{enumerate}{1}
\setlist[ListOfReports]{
    leftmargin=*,
    label={Report \Alph*},
    ref={Report \Alph*},
    resume=ListOfReports,
    itemsep=0em,
    first=\raggedright
  }

\newlist{ListOfArtifacts}{enumerate}{1}
\setlist[ListOfArtifacts]{
    leftmargin=*,
    label={Artifact \Alph*},
    ref={Artifact \Alph*},
    resume=ListOfArtifacts,
    itemsep=0em,
    first=\raggedright
  }

\newlist{ListOfDatasets}{enumerate}{1}
\setlist[ListOfDatasets]{
    leftmargin=*,
    label={Dataset \Alph*},
    ref={Dataset \Alph*},
    resume=ListOfDatasets,
    itemsep=0em,
    first=\raggedright
  }

%%% Potential future lists that might be added:
% ListOfExhibitions
% ListOfDesigns
% ListOfPerformances
% ListOfCompositions
% Although each of these might be treated as an artifact, some fields might have this as a distinct category - hence, they might warrant being separate lists.


% to define a command\B to bold font entries in a table
% based on https://tex.stackexchange.com/questions/469559/bold-entries-in-table-with-s-column-type
\usepackage{etoolbox}
\renewcommand{\bfseries}{\fontseries{b}\selectfont}
\robustify\bfseries
\newrobustcmd{\B}{\bfseries}
