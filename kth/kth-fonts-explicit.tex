%\NeedsTeXFormat{LaTeX2e}[1999/12/01]
%\ProvidesPackage{kth-fonts}
%    [2025/07/19 v0.1 Provides template-specific fonts needed for kththesis.cls]

% --- Create master flags for font groups ---
\newif\ifusenotoserif
\newif\ifusenotosans
\newif\ifusenotomono

\newif\ifusenotonaskharabic
\newif\ifusenotosansarabic

\defaultfontfeatures{Ligatures={TeX}} % This enables TeX style ligatures such as ---, '', ``, and so on

% If bidirectional printing is needed, enable bidi
\ifenabledbidi
    \ifinswedish
        \usepackage[english, main=swedish, provide+=*, bidi=basic]{babel}
    \else
        \usepackage[swedish, main=english, provide+=*, bidi=basic]{babel}
    \fi
\else
    \ifinswedish
        \usepackage[english, main=swedish, provide+=*]{babel}
    \else
        \usepackage[swedish, main=english, provide+=*]{babel}
    \fi
\fi

%% Set up the default serif font
%\babelfont{rm}{TeX Gyre Termes}
%\DeclareFontShape{TU}{TeXGyreTermes(0)}{md}{n}{<->sub * TeXGyreTermes(0)/m/n}{}
%\DeclareFontShape{TU}{TeXGyreTermes(0)}{sb}{n}{<->ssub * TeXGyreTermes(0)/b/n}{}
%\DeclareFontShape{TU}{TeXGyreTermes(0)}{sb}{it}{<->ssub * TeXGyreTermes(0)/b/it}{}
\babelfont{rm}[
  Path = /usr/local/texlive/2024/texmf-dist/fonts/opentype/public/tex-gyre/,  % Specify the directory
  Extension = .otf,             % Specify the file extension
  UprightFont    = *-regular,
  BoldFont       = *-bold,
  ItalicFont     = *-italic,
  BoldItalicFont = *-bolditalic
]{texgyretermes} % Use the base filename here


%% Set up the default sans serif font
%\setsansfont{TeX Gyre Heros}   %% Helvetica like font
%\babelfont{sf}{TeX Gyre Heros}
\babelfont{sf}[
  Path = /usr/local/texlive/2024/texmf-dist/fonts/opentype/public/tex-gyre/,  % Specify the directory
  Extension = .otf,             % Specify the file extension
  UprightFont    = *-regular,
  BoldFont       = *-bold,
  ItalicFont     = *-italic,
  BoldItalicFont = *-bolditalic
]{texgyreheros}

%\setmonofont[Ligatures={NoCommon}, Numbers={Lining,Monospaced}]{TeX Gyre Cursor}  %% Courier like font
%\babelfont{tt}{TeX Gyre Cursor}
\babelfont{tt}[
  Path = /usr/local/texlive/2024/texmf-dist/fonts/opentype/public/tex-gyre/,  % Specify the directory
  Extension = .otf,             % Specify the file extension
  UprightFont    = *-regular,
  BoldFont       = *-bold,
  ItalicFont     = *-italic,
  BoldItalicFont = *-bolditalic
]{texgyrecursor}
\begin{comment}
    \babelfont[english]{rm}{TeX Gyre Termes}
    \babelfont[english]{sf}{TeX Gyre Heros}
    \babelfont[english]{tt}{TeX Gyre Cursor}
\end{comment}

%% set up math font
%  \setmathfont{TeX Gyre Termes Math} %% a math font
\usepackage{mathtools}
\usepackage[warnings-off={mathtools-colon,mathtools-overbracket}]{unicode-math}
% The [version=bold, FakeBold=1.2] is to avoid a warning about the lack of a bold font
%\setmathfont{TeX Gyre Pagella Math}[version=bold, FakeBold=1.2] %% a font for math
\setmathfont{STIX Two Math}[version=normal]
%\setmathfont{TeX Gyre Pagella Math}[version=normal]
%\setmathfont{TeX Gyre Pagella Math}[version=bold, BoldFeatures={FakeBold=1.5}]
  % For both XeLaTeX and LuaLatex for getting access to unicode symbols
  %\newfontfamily\myfont[CharacterVariant=1]{NewCM10-Regular.otf}
  % STIX Project (Scientific and Technical Information Exchange)
  % STIX Two Math does not have bold face - so we fake it
  %\newfontfamily\mystixmathfont[BoldFeatures={FakeBold=1.5}, BoldItalicFeatures={FakeBold=1.5}]{STIX Two Math}
%\newfontfamily\mystixmathfont{STIX Two Math}
\newfontfamily\mystixmathfont{stixtwomath}
  % STIX Two Math does not have a bold font, but it has bold symbols with an without serifs - but you manually have to use them, unless you are in math mode - then you can use \symbf{}
  % and this will return the bold serif version of the character
%\DeclareFontShape{TU}{STIXTwoMath(0)}{b}{n}{<->ssub * STIXTwoMath(0)/m/n}{}
%\DeclareFontShape{TU}{STIXTwoMath(0)}{sb}{n}{<->ssub * STIXTwoMath(0)/m/n}{}
%\newfontfamily\mystixtextfont{STIX Two Text}
\newfontfamily\mystixtextfont{stixtwotext}

% use english as a fallback when in other languages
\babelprovide[import, onchar=ids fonts]{english}

    
  % for new KTH cover
  % Load the Figtree font as it is used for the new KTH graphical profile
  % 

\newfontfamily{\FigtreeFont}[Ligatures=TeX,
        Path=./Figtree/static/,
        Extension = .ttf,
        UprightFont=*-Regular,
        BoldFont=*-Bold,
        BoldItalicFont=*-BoldItalic,
        ItalicFont=*-Italic,
        %FontFace={l}{n}{*-Light},
        %FontFace={l}{it}{*-LightItalic},
        FontFace={md}{n}{*-Medium},
        FontFace={md}{it}{*-MediumItalic},
        FontFace={sb}{n}{*-Semibold},
        FontFace={sb}{it}{*-SemiBoldItalic},
        %FontFace={k}{n}{*-Black},
        %FontFace={k}{it}{*-BlackItalic},
        %FontFace={eb}{n}{Font=*-ExtraBold},
        %FontFace={eb}{it}{Font=*-ExtraBoldItalic}
        ]{Figtree}


\newfontfamily\pageNumberFont{Figtree} %% set the font to use for page numbering

\newfontfamily{\NotoEmojiFont}[Ligatures=TeX,
    Path=./Noto_Emoji/static/,
    Extension = .ttf,
    UprightFont=*-Regular,
    BoldFont=*-Bold,
    %FontFace={l}{n}{*-Light.ttf},
    FontFace={md}{n}{*-Medium},
    FontFace={sb}{n}{*-SemiBold},
    ]{NotoEmoji}

   % To set the abstract headings in Figtree we redefine the abstravt environment to look at the language being used and use the appropriate font, with the default being Figtree
   % The languages that are automatically introduced by Babel have a name of the form xxxfont and xxxfontsf; where xxxfont is the serif font and xxxfontsf is the ssans erif font.
   % This means that for each language that Figtree does not support, you have to define the sans serif and serif font to use.

\ifenabledgreekfont
    \usenotoseriftrue\usenotosanstrue\usenotomonotrue
\fi
\ifenabledcyrillicfont
    \usenotoseriftrue\usenotosanstrue\usenotomonotrue
\fi
\ifenabledvietnamesefont
    \usenotoseriftrue\usenotosanstrue\usenotomonotrue
\fi


\ifusenotoserif
    \babelprovide[import, onchar=ids fonts]{greek}
    \babelprovide[import, onchar=ids fonts]{russian}
    \babelprovide[import, onchar=ids fonts]{ukrainian}
    \babelprovide[import, onchar=ids fonts]{vietnamese}
    %\babelfont[greek, russian, ukrainian, vietnamese]{rm}{Noto Serif}
    % --- Load Noto Serif manually to ensure correct width (avoiding the semicondensed versions) ---
    \babelfont[greek, russian, vietnamese]{rm}[
    % Assuming the fonts are in a place TeX can find them,
    Path = /usr/local/texlive/2024/texmf-dist/fonts/truetype/google/noto/,
    Extension = .ttf,
    UprightFont = NotoSerif-Regular,
    BoldFont = NotoSerif-Bold,
    ItalicFont = NotoSerif-Italic,
    BoldItalicFont = NotoSerif-BoldItalic
]{notoserif} % Use a custom internal name     
\fi
\ifusenotosans
    %\babelfont[greek, russian, ukrainian, vietnamese]{sf}{Noto Sans}
    % --- Load Noto Sans manually to ensure correct width ---
\babelfont[greek, russian, vietnamese]{sf}[
    Path = /usr/local/texlive/2024/texmf-dist/fonts/truetype/google/noto/,
    Extension = .ttf,
    UprightFont = NotoSans-Regular,
    BoldFont = NotoSans-Bold,
    ItalicFont = NotoSans-Italic,
    BoldItalicFont = NotoSans-BoldItalic
]{notosans}
\fi
\ifusenotomono
    \babelfont[greek, russian, ukrainian, vietnamese]{tt}[
    Path=/usr/share/fonts/truetype/noto/,
     Extension = .ttf,
    UprightFont = NotoMono-Regular,
    ]{notomono}
\fi

\ifenableddevanagarifont
\babelprovide[import, onchar=ids fonts]{hindi}
%\babelfont[hindi]{rm}{Noto Serif Devanagari}
%\babelfont[hindi]{sf}{Noto Sans Devanagari}
%\babelfont[hindi]{tt}{Noto Sans Devanagari} % Noto Mono does not have the glyphs
\babelfont[hindi]{rm}«
    Path=/usr/share/fonts/truetype/noto/,
    Extension = .ttf,
    UprightFont = *-Regular,
    BoldFont = *-Bold,
    ItalicFont = *-Italic,
    BoldItalicFont = *-BoldItalic
]{notoserifdevanagari}
\babelfont[hindi]{sf}[
    Path=/usr/share/fonts/truetype/noto/,
    Extension = .ttf,
    UprightFont = *-Regular,
    BoldFont = *-Bold,
    ItalicFont = *-Italic,
    BoldItalicFont = *-BoldItalic
]{notosansdevanagari}
\babelfont[hindi]{tt}[
    Path=/usr/share/fonts/truetype/noto/,
    Extension = .ttf,
    UprightFont = *-Regular,
    BoldFont = *-Bold,
    ItalicFont = *-Italic,
    BoldItalicFont = *-BoldItalic
]{notosansdevanagari} 
\fi

\ifenabledchinesesimplifiedfont
\babelprovide[import, onchar=ids fonts]{chinese-simplified}
%\babelfont[chinese-simplified]{rm}{Noto Serif CJK SC}
%\babelfont[chinese-simplified]{sf}{Noto Sans CJK SC}
%\babelfont[chinese-simplified]{tt}{Noto Sans Mono CJK SC}
\babelfont[chinese-simplified]{rm}[
    Path=/usr/share/fonts/opentype/noto/,
    Extension = .ttc,
    UprightFont = NotoSerifCJK-Regular,
    BoldFont = NotoSerifCJK-Bold,
]{notoserifcjksc}
\babelfont[chinese-simplified]{sf}[
    Path=/usr/share/fonts/opentype/noto/,
    Extension = .ttc,
    UprightFont = NotoSansCJK-Regular,
    BoldFont = NotoSansCJK-Bold,
]{notosanscjksc}
\babelfont[chinese-simplified]{tt}[
    Path=/usr/share/fonts/opentype/noto/,
    Extension = .ttc,
    UprightFont = NotoSansCJK-Regular,
    BoldFont = NotoSansCJK-Bold,
]{notosansmonocjkhk}
\fi

\begin{comment}  
\babelfont[chinese-traditional]{rm}{Noto Serif CJK TC}
\babelfont[chinese-traditional]{sf}{Noto Sans CJK TC}
\babelfont[chinese-traditional]{tt}{Noto Sans Mono CJK TC}
\end{comment} 

\ifenabledjapanesefont
\babelprovide[import, onchar=ids fonts]{japanese}
%\babelfont[japanese]{rm}{Noto Serif CJK JP}
%\babelfont[japanese]{sf}{Noto Sans CJK JP}
%\babelfont[japanese]{tt}{Noto Sans Mono CJK JP}
\babelfont[japanese]{rm}[
    Path=/usr/share/fonts/opentype/noto/,
    Extension = .ttc,
    UprightFont = NotoSerifCJK-Regular,
    BoldFont = NotoSerifCJK-Bold,
]{notoserifcjkjp}
\babelfont[japanese]{sf}[
    Path=/usr/share/fonts/opentype/noto/,
    Extension = .ttc,
    UprightFont = NotoSansCJK-Regular,
    BoldFont = NotoSansCJK-Bold,
]{notosanscjkjp}
\babelfont[japanese]{tt}[
    Path=/usr/share/fonts/opentype/noto/,
    Extension = .ttc,
    UprightFont = NotoSansCJK-Regular,
    BoldFont = NotoSansCJK-Bold,
]{notosansmonocjkjp}
\fi
  
  % If you are going to use Arabic
\ifenabledarabicfont
    \usenotonaskharabictrue
    \usenotosansarabictrue
\fi
\ifusenotonaskharabic
    \babelprovide[import, onchar=ids fonts]{arabic}
    \babelprovide[import, onchar=ids fonts]{centralkurdish}
    %\babelfont[arabic, centralkurdish]{rm}{Noto Naskh Arabic}
    \babelfont[arabic, centralkurdish]{rm}[
    Path=/usr/share/fonts/truetype/noto/,
    Extension = .ttf,
    UprightFont = NotoSansArabic-Regular,
    BoldFont = NotoSansArabic-Bold,    
    ]{notonaskharabic}
\fi
\ifusenotosansarabic
    %\babelfont[arabic, centralkurdish]{sf}{Noto Sans Arabic}
    \babelfont[arabic, centralkurdish]{sf}[
    Path=/usr/share/fonts/truetype/noto/,
    Extension = .ttf,
    UprightFont = NotoNaskhArabic-Regular,
    BoldFont = NotoNaskhArabic-Bold, 
    ]{notonaskharabic}
    %\babelfont[arabic, centralkurdish]{tt}{Noto Sans Arabic}
    \babelfont[arabic, centralkurdish]{tt}[
        Path=/usr/share/fonts/truetype/noto/,
    Extension = .ttf,
    UprightFont = NotoNaskhArabic-Regular,
    BoldFont = NotoNaskhArabic-Bold, 
    ]{notonaskharabic}
\fi
    % If one really needs a monospaced font, one might try Kawkab Mono
    % However, it seems that it is a work in progress - see https://makkuk.com/kawkab-mono/ and https://github.com/aiaf/kawkab-mono/tree/master

%\babelprovide[import, onchar=ids fonts]{centralkurdish}
%\babelfont[centralkurdish]{rm}{Noto Naskh Arabic}
%\babelfont[centralkurdish]{sf}{Noto Sans Arabic}
%\babelfont[centralkurdish]{tt}{Noto Sans Arabic}

      
  % If you are going to use Hebrew or Yiddish
\ifenabledhebrewfont
\babelprovide[import, onchar=ids fonts]{hebrew}
\babelprovide[import, onchar=ids fonts]{yiddish}
%\babelfont[hebrew, yiddish]{rm}{Noto Serif Hebrew}
%\babelfont[hebrew, yiddish]{sf}{Noto Sans Hebrew}
%\babelfont[hebrew, yiddish]{tt}{Noto Sans Hebrew}
\babelfont[hebrew, yiddish]{rm}[
    Path=/usr/share/fonts/truetype/noto/,
    Extension = .ttf,
    UprightFont = NotoSerifHebrew-Regular,
    BoldFont = NotoSerifHebrew-Bold, 
]{notoserifhebrew}
\babelfont[hebrew, yiddish]{sf}[
    Path=/usr/share/fonts/truetype/noto/,
    Extension = .ttf,
    UprightFont = NotoSansHebrew-Regular,
    BoldFont = NotoSansHebrew-Bold, 
]{notosanshebrew}
\babelfont[hebrew, yiddish]{tt}[
    Path=/usr/share/fonts/truetype/noto/,
    Extension = .ttf,
    UprightFont = NotoSansHebrew-Regular,
    BoldFont = NotoSansHebrew-Bold, 
]{notosanshebrew}
\fi

%\babelprovide[import, onchar=ids fonts]{vietnamese}
%\babelfont[vietnamese]{rm}{Noto Serif}
%\babelfont[vietnamese]{sf}{Noto Sans}
%\babelfont[vietnamese]{tt}{Noto Mono}
  
    % The Overleaf TeX Live includes these fonts, so there is little you have to do!
    % The list of such fonts is at https://www.overleaf.com/learn/latex/Questions%2FWhich_OTF_or_TTF_fonts_are_supported_via_fontspec%3F
    %
%\newfontfamily{\NotoSansJPFont}[Ligatures=TeX]{Noto Sans Mono CJK JP Regular}
\newfontfamily{\NotoSansJPFont}[Ligatures=TeX]{notosansmonocjkjp}

\newfontfamily{\NotoSansFont}[Ligatures=TeX,
    Path = /usr/local/texlive/2024/texmf-dist/fonts/truetype/google/noto/,
    Extension = .ttf,
    UprightFont = NotoSans-Regular,
    BoldFont = NotoSans-Bold,
    ItalicFont = NotoSans-Italic,
    BoldItalicFont = NotoSans-BoldItalic
    ]{NotoSansCustomA}
    
\newfontfamily{\NotoSerifFont}[Ligatures=TeX,
    Path = /usr/local/texlive/2024/texmf-dist/fonts/truetype/google/noto/,
    Extension = .ttf,
    UprightFont = NotoSerif-Regular,
    BoldFont = NotoSerif-Bold,
    ItalicFont = NotoSerif-Italic,
    BoldItalicFont = NotoSerif-BoldItalic
    ]{NotoSerifCustomA}

\newfontfamily{\DejaVuSansFont}[Ligatures=TeX]{DejaVu Sans}

% Large expl3 block - for deal with missing characters in Hebrew fonts for Arabic numbers and ASCII punctuation
\makeatletter  % enable @ so we can get value of \f@family
\ExplSyntaxOn
\tl_new:N \l_kththesis_family_name_tl
\tl_new:N \l_kththesis_rmdefault_tl
\tl_new:N \l_kththesis_sfdefault_tl
\tl_new:N \l_kththesis_ttdefault_tl
\bool_new:N \l_kththesis_char_processed
\int_new:N \l_kththesis_char_code_int
\int_new:N \l_kththesis_hebrew_start_int % Declare the integer variable
\int_new:N \l_kththesis_hebrew_end_int


% --- Helper Expl3 Macro to Dynamically Select Fallback Font (for \newunicodechar) ---
% #1 = The Unicode character to output (e.g., "0", ":", "א")
\newcommand{\__kththesis_select_dynamic_fallback_font:n}[1]{%
    \bool_gset_false:N \l_kththesis_char_processed
    
    \int_set:Nn \l_kththesis_char_code_int { \numexpr`#1\relax }
    %\typeout{~~~~--~~kththesis_select_dynamic_fallback_font~\tl_use:N #1~\int_use:N \l_kththesis_char_code_int}
    % Get the current language name
    \tl_set:Nx \l_tmpb_tl { \languagename } 

    % Get the current font's family name and strip the (number) suffix
    \tl_set:Nx \l_kththesis_family_name_tl \f@family
    \regex_replace_once:nnN { \A ( .* ) \( \d+ \) \Z } { \1 } \l_kththesis_family_name_tl % Remove (number)
    %\typeout{~--~Final~stripped~l_fontspec_family_tl:~\tl_use:N \l_kththesis_family_name_tl}
 
    % for \rmdefault
    \tl_set:Nx \l_kththesis_rmdefault_tl \rmdefault
    \regex_replace_once:nnN { \A ( .* ) \( \d+ \) \Z } { \1 } \l_kththesis_rmdefault_tl % Remove (number)
    %\typeout{~--~Final~stripped~rmdefault:~\tl_use:N \l_kththesis_rmdefault_tl}
    
    % for \sfdefault
    \tl_set:Nx \l_kththesis_sfdefault_tl  \sfdefault
    \regex_replace_once:nnN { \A ( .* ) \( \d+ \) \Z } { \1 } \l_kththesis_sfdefault_tl % Remove (number)
    %\typeout{~--~Final~stripped~sfdefault:~\tl_use:N \l_kththesis_sfdefault_tl}

    % for \ttdefault
    \tl_set:Nx \l_kththesis_ttdefault_tl   \ttdefault
    \regex_replace_once:nnN { \A ( .* ) \( \d+ \) \Z } { \1 } \l_kththesis_ttdefault_tl % Remove (number)
    %\typeout{~--~Final~stripped~ttdefault:~\tl_use:N \l_kththesis_ttdefault_tl}

    % Based on the current language and stripped font family, select fallback font
    \str_case:VnF \l_tmpb_tl % Check current language
    {
        {hebrew} { % If the language is Hebrew, prioritize Hebrew fallback fonts for characters.
                \str_if_eq:VVTF {\l_kththesis_family_name_tl}{\l_kththesis_rmdefault_tl}{ \hebrewfontFallback #1 }{}
                \str_if_eq:VVTF {\l_kththesis_family_name_tl}{\l_kththesis_sfdefault_tl}{ \hebrewfontsfFallback #1 }{}
                \str_if_eq:VVTF {\l_kththesis_family_name_tl}{\l_kththesis_ttdefault_tl}{ \hebrewfontsfFallback #1}{} % Fallback to sans  as there is no Noto Mono for Hebrew 
                \bool_gset_true:N \l_kththesis_char_processed
                }
        % Add other language-specific fallback logic here if needed (e.g., for Arabic numbers/symbols if needed)
    }
    { % Default case for all other languages (Latin, Cyrillic, Greek, etc.)
      % Use the general Latin/Cyrillic fallback fonts.
      % Compare the current character with the Hebrew block range
      %\typeout{~~~~--~Check~ if~character~is~IN~range~\tl_use:N #1}
      % Set the range of values
      \int_set:Nn \l_kththesis_hebrew_start_int { "0590 } % U+0590 
      \int_decr:N \l_kththesis_hebrew_start_int
      \int_set:Nn \l_kththesis_hebrew_end_int { "05FF }   % U+05FF
      \int_incr:N \l_kththesis_hebrew_end_int
            % \bool_lazy_and:nTF checks if (#1 > min-1) AND (#1 < max+1)
            \bool_lazy_and:nnT
                { \int_compare_p:nNn { \l_kththesis_char_code_int } > { \l_kththesis_hebrew_start_int  } }
                { \int_compare_p:nNn { \l_kththesis_char_code_int } < { \l_kththesis_hebrew_end_int  } }
            { % True branch: Item is within range replace it
                %\typeout{~~~~--~Character~is~IN~range~\tl_use:N #1}
                \str_if_eq:VVTF {\l_kththesis_family_name_tl}{\l_kththesis_rmdefault_tl}{ \hebrewfontFallback #1 }{}
                \str_if_eq:VVTF {\l_kththesis_family_name_tl}{\l_kththesis_sfdefault_tl}{ \hebrewfontsfFallback #1 }{}
                \str_if_eq:VVTF {\l_kththesis_family_name_tl}{\l_kththesis_ttdefault_tl}{ \hebrewfontsfFallback #1}{} % Fallback to sans as there is no Noto Mono for Hebrew 
                \bool_gset_true:N \l_kththesis_char_processed
            }
        % Set the range of values Alphabetic Presentation Forms
        \int_set:Nn \l_kththesis_start_int { "FB00 } % U+FB00
        \int_decr:N \l_kththesis_start_int
        \int_set:Nn \l_kththesis_end_int { "FB4F }   % U+FB4F
        \int_incr:N \l_kththesis_end_int
            % \bool_lazy_and:nTF checks if (#1 > min-1) AND (#1 < max+1)
            \bool_lazy_and:nnT
                { \int_compare_p:nNn { \l_kththesis_char_code_int } > { \l_kththesis_start_int  } }
                { \int_compare_p:nNn { \l_kththesis_char_code_int } < { \l_kththesis_end_int  } }
            { % True branch: Item is within range replace it
                %\typeout{~~~~--~Character~is~IN~range~\tl_use:N #1}
                \str_if_eq:VVTF {\l_kththesis_family_name_tl}{\l_kththesis_rmdefault_tl}{ \hebrewfontFallback #1 }{}
                \str_if_eq:VVTF {\l_kththesis_family_name_tl}{\l_kththesis_sfdefault_tl}{ \hebrewfontsfFallback #1 }{}
                \str_if_eq:VVTF {\l_kththesis_family_name_tl}{\l_kththesis_ttdefault_tl}{ \hebrewfontsfFallback #1}{} % Fallback to sans as there is no Noto Mono for Hebrew 
                \bool_gset_true:N \l_kththesis_char_processed
            }


    }
}
\ExplSyntaxOff % Deactivate expl3 syntax
\makeatother
