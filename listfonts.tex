\documentclass{article}
\usepackage{luacode}
\newcommand{\dname}[1]{\textbf{#1}}
\newcommand{\fname}[1]{\texttt{#1}}

\title{List all of the available fonts and put them in the font cache}

\author{Gerald Q. Maguire Jr.}
\date{August 2025}

\begin{document}
% based on https://tex.stackexchange.com/a/14171/9075
% addaswyd yn ateb: https://tex.stackexchange.com/a/737051/
% ac: https://tex.stackexchange.com/a/737052/
\maketitle
% Extended on 2025-08-01 by GQMJr with more comments and headings, as well as an exclusion_list (really a Lua table).

A side effect of compiling this file is to create entries for all of the available fonts (other than those in the exclusion list) in the font cache.

Note that font files, such as \fname{NotoSansCJK-Regular.ttc} are a TrueType Collection (TTC), i.e., a file format that combines multiple TrueType or OpenType fonts into a single file. The following are all in a single file:
\begin{itemize}
    \item Noto Sans CJK JP
    \item Noto Sans CJK KR
    \item Noto Sans CJK SC
    \item Noto Sans CJK TC
    \item Noto Sans CJK HK
\end{itemize}
\begin{luacode}
    maxfonts=1000 -- an arbitrary limit on the number of fonts to output information for
    
    -- some fonts lack the glyphs to print their name
    local exclusion_list = {
    ["NotoColorEmoji.ttf"] = true,
    ["NotoEmoji-Regular.ttf"] = true,
    ["NotoSansGothic-Regular.ttf"] = true,
    ["NotoSansLinearA-Regular.ttf"] = true,
    ["NotoSansLinearB-Regular.ttf"] = true,
    ["NotoKufiArabic-Black.ttf"] = true,
    ["NotoKufiArabic-Bold.ttf"] = true,
    ["NotoKufiArabic-Light.ttf"] = true,
    ["NotoKufiArabic-Medium.ttf"] = true,
    ["NotoKufiArabic-Regular.ttf"] = true,
    ["NotoKufiArabic-SemiBold.ttf"] = true,
    ["NotoKufiArabic-Thin.ttf"] = true,
    ["NotoRashiHebrew-Black.ttf"] = true,
    ["NotoRashiHebrew-Bold.ttf"] = true,
    ["NotoRashiHebrew-Light.ttf"] = true,
    ["NotoRashiHebrew-Medium.ttf"] = true,
    ["NotoRashiHebrew-Regular.ttf"] = true,
    ["NotoRashiHebrew-SemiBold.ttf"] = true,
    ["NotoRashiHebrew-Thin.ttf"] = true,
    ["NotoNastaliqUrdu-Bold.ttf"] = true,
    ["NotoNastaliqUrdu-Regular.ttf"] = true,
    ["NotoMusic-Regular.ttf"] = true,
    ["NotoSansDevanagari-Medium.ttf"] = true,
    ["NotoSerifDisplay-Thin.ttf"] = true,
    ["NotoSerifHebrew-Bold.ttf"] = true,
    ["NotoSansArabic-Thin.ttf"] = true,
    ["NotoSansHebrew-Bold.ttf"] = true,
    ["NotoSansHebrew-Regular.ttf"] = true,
    ["NotoSerifDevanagari-Light.ttf"] = true,
    ["NotoSansArabic-Regular.ttf"] = true,
    ["NotoSerifHebrew-Thin.ttf"] = true,
    ["NotoSansArabic-SemiBold.ttf"] = true,
    ["NotoSansDevanagari-Bold.ttf"] = true,
    ["NotoSerifHebrew-Medium.ttf"] = true,
    ["NotoSerifDevanagari-Bold.ttf"] = true,
    ["NotoSerifDevanagari-SemiBold.ttf"] = true,
    ["NotoSansHebrew-Thin.ttf"] = true,
    ["NotoNaskhArabic-Medium.ttf"] = true,
    ["NotoSansDevanagari-Regular.ttf"] = true,
    ["NotoSansHebrew-SemiBold.ttf"] = true,
    ["NotoNaskhArabic-SemiBold.ttf"] = true,
    ["NotoSansHebrew-Light.ttf"] = true,
    ["NotoSansArabic-Light.ttf"] = true,
    ["NotoSerifDevanagari-Medium.ttf"] = true,
    ["NotoSerifDevanagari-Thin.ttf"] = true,
    ["NotoSerifHebrew-Regular.ttf"] = true,
    ["NotoSansDevanagari-Thin.ttf"] = true,
    ["NotoSansDevanagari-SemiBold.ttf"] = true,
    ["NotoSansArabic-Bold.ttf"] = true,
    ["NotoSansArabic-Medium.ttf"] = true,
    ["NotoSerifHebrew-SemiBold.ttf"] = true,
    ["NotoSerifDevanagari-Regular.ttf"] = true,
    ["NotoNaskhArabic-Regular.ttf"] = true,
    ["NotoNaskhArabic-Bold.ttf"] = true,
    ["NotoSansDevanagari-Light.ttf"] = true,
    ["NotoSansHebrew-Medium.ttf"] = true,
    ["NotoSerifHebrew-Light.ttf"] = true,


    -- math and symbol fonts
    ["NotoSansSymbols2-Regular.ttf"] = true,
    ["STIXSizeTwoSym-Regular.otf"] = true,
    ["STIXIntegralsUp-Bold.otf"] = true,
    ["STIXSizeThreeSym-Bold.otf"] = true,
    ["STIXIntegralsUpD-Regular.otf"] = true,
    ["STIXSizeFourSym-Regular.otf"] = true,
    ["STIXIntegralsD-Bold.otf"] = true,
    ["STIXSizeThreeSym-Regular.otf"] = true,
    ["STIXNonUnicode-Bold.otf"] = true,
    ["STIXIntegralsD-Regular.otf"] = true,
    ["STIXVariants-Regular.otf"] = true,
    ["STIXVariants-Bold.otf"] = true,
    ["STIXIntegralsUpSm-Regular.otf"] = true,
    ["STIXIntegralsSm-Regular.otf"] = true,
    ["STIXSizeTwoSym-Bold.otf"] = true,
    ["STIXIntegralsUpD-Bold.otf"] = true,
    ["STIXSizeOneSym-Regular.otf"] = true,
    ["STIXIntegralsSm-Bold.otf"] = true,
    ["STIXIntegralsUp-Regular.otf"] = true,
    ["STIXNonUnicode-Italic.otf"] = true,
    ["STIXIntegralsUpSm-Bold.otf"] = true,
    ["STIXSizeOneSym-Bold.otf"] = true,
    ["STIXNonUnicode-Regular.otf"] = true,
    ["STIXSizeFiveSym-Regular.otf"] = true,
    ["STIXNonUnicode-BoldItalic.otf"] = true,
    ["STIXSizeFourSym-Bold.otf"] = true,


    -- You can add any other font basenames here
    }
-- The following fonts are missing a period
-- NotoSansSymbolsBold�ttf
-- NotoSansSymbolsLight�ttf

    -- use the API to get the font index
  myfonts = luaotfload.aux.read_font_index()

  -- Get the number of font entries from the 'mappings' table
 local number_of_fonts = #myfonts.mappings

  tex.print("\\section*{Got all of the " .. number_of_fonts .. " available fonts}")
  
  -- List the font families
  tex.print("\\section{myfonts myfonts.families}")
  t={ myfonts, myfonts.families }
  for _,b in ipairs(t) do
    -- tex.print("\\ttfamily " .. b .. ": \\par")
    for i,j in pairs(b) do
      tex.print("\\ttfamily " .. i .. ":: " .. type(j) .. "\\par")
    end
  end

  
  tex.print("\\section*{myfonts.mappings plainname}")
  n = 0
  for _,i in ipairs(myfonts.mappings) do
    n = n + 1
    if n > maxfonts then break end
    tex.print("\\ttfamily " .. myfonts.mappings[n].plainname .. "\\par")
  end

  
  tex.print("\\section*{myfonts.mappings[1]}")
  for i,_ in pairs(myfonts.mappings[1]) do
    tex.print("\\ttfamily " .. i .. " ")
  end
  tex.print("\\par")

  -- For all of the fonts, output their basename using the font.
  -- Do not do this for those in the exclusion_list (these fonts would produce an error indicating that glyphs were not available).
  tex.print("\\clearpage")
  tex.print("\\section*{myfonts.mappings basename}")
  n=0
  for _,i in ipairs(myfonts.mappings) do
    n = n + 1
    if n > maxfonts then return 0 end
    if exclusion_list[myfonts.mappings[n].basename] then
    tex.print("\\ttfamily "  .. n .. " " ..(string.gsub(myfonts.mappings[n].basename,"_","\\_")) .. "\\font\\myfont = {file:" .. myfonts.mappings[n].basename .. "} at 14pt \\myfont " .. " " .. "\\par")
--   elseif n == 112 or n == 246 or (255 < n) or (106 < n and n < 109) or n == 254 then
--   tex.print("\\ttfamily "  .. n .. " " ..(string.gsub(myfonts.mappings[n].basename,"_","\\_")) .. "\\par")
    else
    tex.print("\\ttfamily "  ..
      "\\font\\myfont = {file:" .. myfonts.mappings[n].basename .. "} at 14pt \\myfont " .. n .. " " ..
      (string.gsub(myfonts.mappings[n].basename,"_","\\_")) ..
    "\\par")
    end
  end
\end{luacode}
\end{document}
